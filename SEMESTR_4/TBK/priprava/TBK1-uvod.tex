\documentclass[a4paper,12pt]{article}
\usepackage[czech]{babel}
\usepackage[utf8]{inputenc}
\usepackage{amsmath}
\usepackage{amsfonts}

\title{TBK1 - úvod}
\begin{document}

\maketitle
\newpage
\section*{1. Příklady běžně používaných bezdrátových komunikačních systémů a jejich pracovní frekvenční pásma}

\begin{itemize}
    \item \textbf{GSM} – 900~MHz a 1800~MHz
    \item \textbf{Wi-Fi} – 2{,}4~GHz / 5~GHz
    \item \textbf{5G komunikace} – 27–80~GHz
    \item \textbf{FM rádio} – 87{,}5–108~MHz
    \item \textbf{Bluetooth} – 2{,}4~GHz
\end{itemize}

\section*{2. Aplikace, kde je použití bezdrátových radiových prostředků nezbytné}

\begin{itemize}
    \item Družice
    \item Letadlo
    \item Loď
    \item Auto
    \item Přenosná zařízení
    \item Nebušice
\end{itemize}

\section*{3. Aplikace, kde je použití bezdrátových radiových prostředků velmi výhodné a pohodlné}

\begin{itemize}
    \item Obecně tam, kde nechceme tahat kabeláž (optika, koaxiál...), nemusíme mít povolení na rozkopání ulice, nebo je trasa příliš dlouhá – dva „talíře“ (antény) mohou být levnější než použití bagru.
    \item Mobilní telefon – nemusíme používat telefonní budky nebo telegraf.
\end{itemize}

\section*{4. Výhody použití vyšších pracovních frekvencí}

\begin{itemize}
    \item Volnější frekvenční pásma
    \item Větší šířka pásma \( B \)
    \item Vyšší rychlosti přenosu dat \( C \)
    \item Možnost integrace antény přímo na čip
\end{itemize}

\section*{5. Základní rozměr antén a použití větších/menších antén}

\begin{itemize}
    \item Základní rozměr antény: \( \lambda / 2 \)
    \item \textbf{Větší než} \( \lambda / 2 \): Parabolické antény (např. satelitní)
    \item \textbf{Menší než} \( \lambda / 2 \): Elektronika (např. ESP moduly)
    
    \begin{itemize}
        \item Např. pro frekvenci 2{,}4~GHz platí \( \lambda \approx 15 \,\text{cm} \)
    \end{itemize}
\end{itemize}

\end{document}
